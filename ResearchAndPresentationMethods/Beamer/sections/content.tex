%%%%%%%%%%%%%%%%%%%%%%%%%%%%%%%%%%%%
\framepic[10.8]{images/book}{
 \begin{textblock}{7}(7,2.5)
    {\color{colorblue}\hugetext{\textbf{}}}
 \end{textblock}
}
%%%%%%%%%%%%%%%%%%%%%%%%%%%%%%%%%%%%
	\begin{frame}
		
			\frametitle{THE CODING PROCESS\\
\begin{center}
\begin{quote}
\noindent Our discussion of quantitative content analysis illustrates the difference between qualitative and quantitative content analysis, yet there are also similarities. Despite our con­cern with differentiating between different methods that use the same terms, we are cognizant of the research axiom "there can be no quantification without qualification.\\
\end{quote}
\end{center}
		}
		
	\end{frame}	
%%%%%%%%%%%%%%%%%%%%%%%%%%%%%%%%%%%%%%%
	\begin{frame}
		
\frametitle{THE CODING PROCESS\\
\begin{center}
\begin{quote}
" This expression underlines the necessity of first being able to identify and categorize the variable before we can count, assign a category, or interpret the content. The chal­lenge is twofold: first we must be able to find a commonly understood and easily dis­tinguishable unit of analysis, and, second, we must be able to reliably and consistently classify each of these units
\end{quote}
\end{center}
}
	\end{frame}	
%%%%%%%%%%%%%%%%%%%%%%%%%%%%%%%%%%%%%%%%%
	\begin{frame}

		\frametitle{DEFINING THE UNIT OF ANALYSIS\\
\begin{center}
\begin{quote}
\noindent The following hypothetical example of an email extract among students in an e-learning context provides us with an example of the challenge, that confront the e-rcsearcher engaged in quantitative content analysis.\\
\end{quote}
\end{center}	
}
	\end{frame}	

%%%%%%%%%%%%%%%%%%%%%%%%%%%%%

	\begin{frame}
		

		\frametitle{DEFINING THE UNIT OF ANALYSIS\\
\begin{center}
\begin{quote}
I really hate the way we have to answer ALL THESE QUESTIONS before getting the data right-how about you? I have very little time this week for this assignment. The last time I tried this I got bogged down with chapter 3 do we really have to know about semiotics, I want to get onto the project analysis first then decide if we need all this theory how about I'll do the first question and you guys do the rest.\\
\end{quote}
\end{center}	
}
		
	\end{frame}	
%%%%%%%%%%%%%%%%%%%%%%%%%%%%%

%%%%%%%%%%%%%%%%%%%%%%%%%%%%%%%%%%%%%
%%%%%%%%%%%%%%%%%%%%%%%%%%%%%%%%%%%%%
%%%%%%%%%%%%%%%%%%%%%%%%%%%%%%%%%%%
